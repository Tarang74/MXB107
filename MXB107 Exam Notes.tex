%!TEX TS-program = xelatex
%!TEX options = -aux-directory=Debug -shell-escape -file-line-error -interaction=nonstopmode -halt-on-error -synctex=1 "%DOC%"
\documentclass{article}
\input{LaTeX-Submodule/template.tex}

% Additional packages & macros
\DeclareMathOperator*{\argmax}{arg\,max}

\usepackage{mathdots}
\setitemize{leftmargin=*,topsep=1ex,partopsep=0ex,itemsep=-1ex,partopsep=0ex,parsep=1ex}
\setlist{leftmargin=*,topsep=1ex,partopsep=0ex,itemsep=-1ex,partopsep=0ex,parsep=1ex}

\usepackage{changepage} % Modify page width
\usepackage{multicol} % Use multiple columns
\usepackage[explicit]{titlesec} % Modify section heading styles

\titleformat{\section}{\raggedright\normalfont\bfseries}{}{0em}{#1}
\titleformat{\subsection}{\raggedright\normalfont\small\bfseries}{}{0em}{#1}

%% A4 page
\geometry{
a4paper,
margin = 10mm
}

%% Hide horizontal rule
\renewcommand{\headrulewidth}{0pt}
\fancyhead{}

%% Hide page numbers
\pagenumbering{gobble}

%% Multi-columns setup
\setlength\columnsep{4pt}

%% Paragraph setup
\setlength\parindent{0pt}
\setlength\parskip{0pt}

%% Customise section heading styles
% \titleformat*\section{\raggedright\bfseries}

\begin{document}
% Modify spacing
\titlespacing*\section{0pt}{0.5ex}{1ex}
\titlespacing*\subsection{0pt}{0.5ex}{1ex}
%
\setlength\abovecaptionskip{8pt}
\setlength\belowcaptionskip{-15pt}
\setlength\textfloatsep{0pt}
%
\setlength\abovedisplayskip{1pt}
\setlength\belowdisplayskip{1pt}

\begin{multicols}{3}
    \addcontentsline{toc}{section}{Introduction}
    \subsection{Population}
    The entire group we are concerned with.
    \subsection{Sample}
    A representative subset of a population.
    \subsection{Quantitative Data}
    Numerical data.
    Nominal (discrete or continuous), or ordinal (ordered).
    \subsection{Qualitative Data}
    Categorical data, e.g.\ colour, model.
    \section{Measures of Centrality}
    \subsection{Mean (arithmetic mean/average)}
    Given \(n\) observations \(x_1,\: x_2,\: \ldots,\: x_n\)
    \begin{equation*}
        \frac{1}{n} \sum_{i = 1}^n x_i
    \end{equation*}
    Sample: \(\overline{x}\).
    Population: \(\mu\).
    \subsection{Median (middle value)}
    The mean can be misleading when the data is skewed.
    When arranged from smallest to largest:
    \begin{equation*}
        \text{median} = \begin{cases}
            x^{\left( \frac{n + 1}{2} \right)} & n \text{ odd}
            \\
            \frac{x^{\left( \frac{n}{2} \right)} + x^{\left( \frac{n}{2} + 1 \right)}}{2}
                                               & n \text{ even}
        \end{cases}
    \end{equation*}
    \subsection{Mode (most common value)}
    \section{Measures of Dispersion}
    Variation in a set of observations.
    \subsection{Range}
    Difference between max and min value.
    \subsection{Variance}
    Average squared deviations from mean.
    \textbf{Sample variance:}
    \begin{equation*}
        s^2 = \frac{1}{n - 1} \sum_{i = 1}^n (x_i - \overline{x})^2.
    \end{equation*}
    \textbf{Population variance:}
    \begin{equation*}
        \sigma^2 = \frac{1}{N} \sum_{i = 1}^N (x_i - \mu)^2.
    \end{equation*}
    \begin{equation*}
        \Var{\left( X \right)} = \sigma^2 = \E{\left( X^2 \right)} - \E{\left( X \right)}^2
    \end{equation*}
    \subsection{Standard Deviation}
    \textbf{Population SD:}
    \(
    \sigma = \sqrt{\sigma^2}.
    \) \\
    \textbf{Sample standard deviation:}
    \(
    s = \sqrt{s^2}.
    \)
    \subsection{Chebyshev's Theorem}
    Given \(n\) observations, at least
    \begin{equation*}
        1 - \frac{1}{k^2}
    \end{equation*}
    of them are within \(k \sigma\) of \(\mu\),
    where \(k \geq 1\).
    \subsection{Empirical Rule}
    For unimodal and symmetric data
    \begin{itemize}
        \item 68\% of data falls within \(\sigma\) of \(\mu\)
        \item 95\% of data falls within \(2\sigma\) of \(\mu\)
        \item 99\% of data falls within \(3\sigma\) of \(\mu\)
    \end{itemize}
    Approx. \(\text{range} \approx 4 s\)
    s.t. \(s = \frac{\text{range}}{4}\).
    \section{Skew}
    Asymmetry of the distribution.
    % No calculations needed according to Slack
    % For a finite population of size \(N\), the \textbf{population skew} is defined as
    % \begin{equation*}
    %     \frac{1}{N} \sum_{i = 1}^N \left( \frac{x_i - \mu}{\sigma} \right)^3
    % \end{equation*}
    % For a sample of size \(n\), the \textbf{sample skew} is defined as
    % \begin{equation*}
    %     \frac{\frac{1}{n} \sum_{i = 1}^n \left( x_i - \overline{x} \right)^3}{\left( \sqrt{\frac{1}{n - 1} \sum_{i = 1}^n \left( x_i - \overline{x} \right)^2} \right)^3}
    % \end{equation*}
    \begin{itemize}
        \item \textbf{Positive (right-skewed)} --- ``tail'' of distribution is \textbf{longer on the right}
        \item \textbf{Negative (left-skewed)} --- ``tail'' of distribution is \textbf{longer on the left}
    \end{itemize}
    \section{Measures of Rank}
    \subsection{Z-Score}
    Compare relative rank between members of a population.
    \begin{equation*}
        Z = \frac{x  - \mu}{\sigma} \quad \text{or} \quad \frac{x - \overline{x}}{s}
    \end{equation*}
    \subsection{Quantiles}
    For a set of \(n\) observations, \(x_q\) is the \(q\)-th quantile, if \(q\)\% of the observations are less than \(x_q\).
    \textbf{Inter-Quartile Range}: \(x_{75} - x_{25}\).
    \subsection{Covariance}
    Measure of the linear correlation between variables.
    \begin{equation*}
        s_{xy} = \frac{\sum_{i = 1}^n \left( x_i - \overline{x} \right) \left( y_i - \overline{y} \right)}{n - 1}.
    \end{equation*}
    When \(x = y\), \(s_{xy} = s^2\).
    \begin{itemize}
        \item \(s_{xy} > 0\): As \(x\) increases, \(y\) also increases.
        \item \(s_{xy} < 0\): As \(x\) increases, \(y\) decreases.
        \item \(s_{xy} \approx 0\): No relationship between \(x\) and \(y\).
    \end{itemize}
    \subsection{Correlation Coefficient}
    \begin{equation*}
        -1 \le r_{xy} = \frac{s_{xy}}{s_x s_y} \le 1
    \end{equation*}
    \textbf{No linear relationship} if \(r_{xy} = 0\), but not necessarily \textbf{no relationship}.
    \section{Events and Probability}
    % \subsection{Event}
    % Set of outcomes from an experiment.
    % \subsection{Sample Space}
    % Set of all possible outcomes \(\Omega\).
    % \subsection{Intersection}
    % Outcomes occur in both \(A\) and \(B\)
    % \begin{equation*}
    %     A \cap B \quad\quad \text{or} \quad\quad AB
    % \end{equation*}
    % \subsection{Disjoint}
    % No common outcomes, \(AB = \varnothing\)
    % \begin{equation*}
    %     \Pr{\left( AB \right)} = \Pr{\left( A \,\vert\, B \right)} = 0
    % \end{equation*}
    % \subsection{Union}
    % Set of outcomes in either \(A\) or \(B\)
    % \begin{equation*}
    %     A \cup B
    % \end{equation*}
    % \subsection{Complement}
    % Set of all outcomes not in \(A\), but in \(\Omega \)
    % \begin{align*}
    %     A\overline{A}       & = \varnothing \\
    %     A \cup \overline{A} & = \Omega
    % \end{align*}
    % \subsection{Subset}
    % \(A\) is a (non-strict) subset of \(B\) if all elements in \(A\) are also in \(B\) --- \(A \subset B\).
    % \begin{equation*}
    %     AB = A \quad\quad \text{and} \quad\quad A \cup B = B
    % \end{equation*}
    % \begin{equation*}
    %     \forall A:A\subset \Omega \land \varnothing \subset A
    % \end{equation*}
    % \begin{align*}
    %     \Pr{\left( A \right)}             & \leq \Pr{\left( B \right)}                            \\
    %     \Pr{\left( B \,\vert\, A \right)} & = 1                                                   \\
    %     \Pr{\left( A \,\vert\, B \right)} & = \frac{\Pr{\left( A \right)}}{\Pr{\left( B \right)}}
    % \end{align*}
    % \subsection{Identities}
    % \begin{align*}
    %     A \left( BC \right)            & = \left( AB \right) C                             \\
    %     A \cup \left( B \cup C \right) & = \left( A \cup B \right) \cup C                  \\
    %     A \left(B \cup C\right)        & = AB \cup AC                                      \\
    %     A \cup BC                      & = \left( A \cup B \right) \left( A \cup C \right)
    % \end{align*}
    % \subsection{Probability}
    % Measure of the likeliness of an event occurring
    % \begin{equation*}
    %     \Pr{\left( A \right)} \quad\quad \text{or} \quad\quad \mathrm{P}\left( A \right)
    % \end{equation*}
    % \begin{equation*}
    %     0 \leq \Pr{\left( A \right)} \leq 1
    % \end{equation*}
    % where a probability of 0 never happens, and 1 always happens.
    % \begin{align*}
    %     \Pr{\left( \Omega \right)}       & = 1                         \\
    %     \Pr{\left( \overline{A} \right)} & = 1 - \Pr{\left( A \right)}
    % \end{align*}
    \subsection{Multiplication Rule}
    For independent events \(A\) and \(B\)
    \begin{equation*}
        \Pr{\left( AB \right)} = \Pr{\left( A \right)} \Pr{\left( B \right)}.
    \end{equation*}
    For dependent events \(A\) and \(B\)
    \begin{equation*}
        \Pr{\left( AB \right)} = \Pr{\left( A \,\vert \, B \right)} \Pr{\left( B \right)}
    \end{equation*}
    \subsection{Addition Rule}
    For independent \(A\) and \(B\)
    \begin{equation*}
        \Pr{\left( A \cup B \right)} = \Pr{\left( A \right)} + \Pr{\left( B \right)} - \Pr{\left( AB \right)}.
    \end{equation*}
    % If \(AB = \varnothing \), then \(\Pr{\left( AB \right)} = 0\), so that \(\Pr{\left( A \cup B \right)} = \Pr{\left( A \right)} + \Pr{\left( B \right)}\).
    \subsection{De Morgan's Laws}
    \begin{align*}
        \overline{A \cup B} & = \overline{A} \ \overline{B}     \\
        \overline{AB}       & = \overline{A} \cup \overline{B}.
    \end{align*}
    % \begin{align*}
    %     \Pr{\left( A \cup B \right)} & = 1 - \Pr{\left( \overline{A} \ \overline{B} \right)}    \\
    %     \Pr{\left( AB \right)}       & = 1 - \Pr{\left( \overline{A} \cup \overline{B} \right)}
    % \end{align*}
    \subsection{Bayes' Theorem}
    \begin{equation*}
        \Pr{\left( A \,\vert\, B \right)} = \frac{\Pr{\left( B \,\vert\, A \right)}\Pr{\left( A \right)}}{\Pr{\left( B \right)}}
    \end{equation*}
    % \section{Random Variables}
    % Measurable variable whose value holds some uncertainty.
    % An event is when a random variable assumes a certain value or range of values.
    % \subsection{Probability Distribution}
    % The probability distribution of a random variable \(X\) is a function that links all outcomes \(x \in \Omega\)
    % to the probability that they will occur \(\Pr{\left( X = x \right)}\).
    \subsection{Probability Mass Function}
    \begin{equation*}
        \Pr{\left( X = x \right)} = p_x
    \end{equation*}
    \subsection{Probability Density Function}
    \begin{equation*}
        \Pr{\left( x_1 \leq X \leq x_2 \right)} = \int_{x_1}^{x_2} f\left( x \right) \odif{x}
    \end{equation*}
    \subsection{Cumulative Distribution Function}
    % Probability that a random variable is
    % less than or equal to a particular realisation \(x\).

    % \(F\left( x \right)\) is a valid CDF if:
    % \begin{enumerate}
    %     \item \(F\) is monotonically increasing and continuous
    %     \item \(\lim_{x \to -\infty} F\left( x \right) = 0\)
    %     \item \(\lim_{x \to \infty} F\left( x \right) = 1\)
    % \end{enumerate}
    \begin{equation*}
        \odv{F\left( x \right)}{x} = \odv{}{x} \int_{-\infty}^x f\left( u \right) \odif{u} = f\left( x \right)
    \end{equation*}
    % \subsection{Complementary CDF (Survival Function)}
    % \begin{equation*}
    %     \Pr{\left( X > x \right)} = 1 - \Pr{\left( X \leq x \right)} = 1 - F\left( x \right)
    % \end{equation*}
    % Apparently not on exam says Slack
    % \subsection{\texorpdfstring{\(p\)}{p}-Quantiles}
    % \begin{equation*}
    %     F\left( x \right) = \int_{-\infty}^x f\left( u \right) \odif{u} = p
    % \end{equation*}
    % \subsection{Special Quantiles}
    % \begin{align*}
    %     \text{Lower quartile \(q_1\):}  &  &  & p = \frac{1}{4} \\
    %     \text{Median \(m\):}            &  &  & p = \frac{1}{2} \\
    %     \text{Upper quartile \(q_2\):}  &  &  & p = \frac{3}{4} \\
    %     \text{Interquartile range IQR:} &  &  & q_2 - q_1
    % \end{align*}
    % \subsection{Quantile Function}
    % \begin{equation*}
    %     x = F^{-1}\left( p \right) = Q\left( p \right)
    % \end{equation*}
    % \subsection{Expectation (Mean)}
    % Expectation from \(\infty\) observations. For \(a < c < b\):
    % \begin{equation*}
    %     \E{\left(X\right)} = \begin{aligned}[t]
    %          & -\int_{a}^c F\left( x \right) \odif{x}                     \\
    %          & + \int_c^b \left(1 - F\left( x \right)\right) \odif{x} + c
    %     \end{aligned}
    % \end{equation*}
    % \subsection{Variance}
    % Measure of spread of the distribution (average squared distance of each value from the mean).
    % \subsection{Standard Deviation}
    % \begin{equation*}
    %     \sigma = \sqrt{\Var{\left( X \right)}}
    % \end{equation*}
    \subsection{Central Limit Theorem}
    For a sample of size \(n \ge 30\),
    % from any random probability distribution with expected value \(\mu\)
    % and variance \(\sigma^2\),
    \begin{equation*}
        \frac{\sqrt{n}\left( \overline{x} - \mu \right)}{\sigma} \overset{p}{\rightarrow} \operatorname{N}{\left( 0,\: 1 \right)}
    \end{equation*}
    % meaning that increasing the sample size will lead to a more normal distribution.
    % In this case, a sample size of \(n = 30\) is sufficient to approximate a normal distribution.
    \subsection{Standard Error}
    \begin{equation*}
        \operatorname{SE}{\left( \overline{x} \right)} = \frac{\sigma^2}{n}
    \end{equation*}
    \subsection{Sample Proportion}
    For a sample of size \(n\) if \(x\) members have a particular characteristic:
    \begin{equation*}
        \hat{p} = \frac{x}{n}.
    \end{equation*}
    If \(x\) follows a binomial distribution, then \(\E{\left( x \right)} = np\) and \(\Var{\left( x \right)} = np\left( 1 - p \right)\).
    \begin{equation*}
        \E{\left( \hat{p} \right)} = p
    \end{equation*}
    \begin{equation*}
        \operatorname{SE}{\left( \hat{p} \right)} = \sqrt{\frac{p\left( 1 - p \right)}{n}}.
    \end{equation*}
    For \(np > 5\) and \(n\left( 1 - p \right) > 5\).
    % \subsection{Assessing Normality}
    % \begin{itemize}
    %     \item Histograms: if the data is approximately normal, then the histogram will be approximately symmetric and unimodal.
    %     \item Boxplots: boxplots can be useful for showing outliers and skewness. Extreme clusters of an excessive number of
    %           outliers can be evidence of non-normality.
    %     \item Normal probability plots (\(q\)-\(q\) plots): these plots are constructed by plotting the sorted data values against
    %           their \(Z\)-scores. If the data is approximately normal, then the points will lie approximately on a straight line.
    % \end{itemize}
    \section{Large Sample Estimation}
    % \subsection{Point Estimation}
    \subsection{Moments}
    \begin{align*}
        \mu_n & = \E{\left( X^n \right)} = \int_{-\infty}^\infty x^n f\left( x \right) \odif{x}
    \end{align*}
    for PDF \(f\left( x \right)\), and \(\mu_1 = \E{\left( X \right)}\) and \(\Var{\left( X \right)} = \mu_2 - \mu_1^2\).
    For samples:
    \begin{align*}
        m_n & = \frac{1}{n} \sum_{i = 1}^n x_i^n
    \end{align*}
    where \(\overline{x} = m_1\) and \(s^2 = m_2 - m_1^2\).
    \subsection{Maximum Likelihood Estimation}
    \begin{equation*}
        \mathcal{L}\left( \theta \,\vert\, \symbfit{x} \right) = \prod_{i = 1}^n f\left( x_i \right),\: \hat{\theta} = \argmax_\theta{\mathcal{L}}.
    \end{equation*}
    \begin{equation*}
        \ell\left( \theta \,\vert\, \symbfit{x} \right) = \sum_{i = 1}^n \log{\left( f\left( x_i \right) \right)},\: \hat{\theta} = \argmax_\theta{\ell}.
    \end{equation*}
    \subsection{Bias}
    \begin{equation*}
        \operatorname{Bias}\left( \hat{\theta} \right) = \E{\left( \hat{\theta} \right)} - \theta
    \end{equation*}
    \begin{equation*}
        \E{\left( \hat{\theta} \right)} = \theta \quad \text{\(\hat{\theta}\) unbiased}
    \end{equation*}
    Given \(\hat{\theta}_1\) and \(\hat{\theta}_2\), choose \(\hat{\theta}_1\) over \(\hat{\theta}_2\) if
    \begin{equation*}
        \Var{\left( \hat{\theta}_1 \right)} < \Var{\left( \hat{\theta}_2 \right)}
    \end{equation*}
    \subsection{Mean square error}
    The estimators of \(\theta = \E{\left( X \right)}\)
    are selected to minimise \textbf{mean square error}:
    \begin{align*}
        \operatorname{MSE}\left( \hat{\theta} \right) & = \E{\left( \left( \hat{\theta} - \theta \right)^2 \right)}                             \\
                                                      & = \operatorname{Bias}\left( \hat{\theta} \right)^2 + \Var{\left( \hat{\theta} \right)}.
    \end{align*}
    \subsection{Confidence Intervals}
    For a \textbf{confidence level} of \(\left( 1 - \alpha \right)\%\).
    \begin{align*}
        {CI}_{1 - \alpha} & = \hat{\theta} \pm Z_{\alpha / 2} \operatorname{SE}\left( \hat{\theta} \right)
    \end{align*}
    \section{Hypothesis Testing}
    Define a falsifiable \textbf{null hypothesis} \(H_0\) and reject if the test statistic \(T\left( \symbfit{x} \right)\) lies in the rejection region \(R = \neg CI_{1 - \alpha}\).
    \(R\) satisfies: \(\Pr{\left( T\left( \symbfit{x} \in R \right) \right)} = \alpha\).
    \section{Small Sample Inference}
    When \(n < 30\):
    \begin{equation*}
        T\left( \symbfit{x} \right) \sim t_{\nu,\alpha/2}
    \end{equation*}
    where the degrees of freedom \(\nu = n - 1\).
    \begin{align*}
        \E{\left( X \right)}   & = 0                   \\
        \Var{\left( X \right)} & = \frac{\nu}{\nu - 2} \\
    \end{align*}
    \subsection{Population Mean}
    \begin{equation*}
        {CI}_{1-\alpha} = \overline{x} \pm Z_{\alpha/2} \frac{\sigma}{\sqrt{n}}
    \end{equation*}
    \begin{equation*}
        T\left( \symbfit{x} \right) = \frac{\overline{x} - \mu_0}{\sigma/\sqrt{n}}
    \end{equation*}
    for small samples use \(t_{\nu,\alpha/2}\).
    \subsection{Population Proportion}
    \begin{equation*}
        {CI}_{1-\alpha} = \hat{p} \pm Z_{\alpha/2} \sqrt{\frac{\hat{p}\left( 1 - \hat{p} \right)}{n}}
    \end{equation*}
    \begin{equation*}
        T\left( \symbfit{x} \right) = \frac{\sqrt{n} \left( \hat{p} - p_0 \right)}{\sqrt{p_0 \left( 1 - p_0 \right)}}.
    \end{equation*}
    \(n \hat{p} > 5\) and \(n \left( 1 - \hat{p} \right) > 5\).
    \subsection{Paired Differences}
    For two dependent samples,
    \begin{equation*}
        \overline{d} = \overline{x}_1 - \overline{x}_2.
    \end{equation*}
    \begin{equation*}
        T\left( \symbfit{x} \right) = \frac{\overline{d} - d_0}{s_d / \sqrt{n}}
    \end{equation*}
    \subsection{Difference of Population Means}
    % Given two population means \(\mu_1\) and \(\mu_2\), we expect the difference of the population means
    % and the sample means to be equal. Consider the expectation of the difference of the sample means:
    % \begin{equation*}
    %     \E{\left( \overline{x}_1 - \overline{x}_2 \right)} = \mu_1 - \mu_2
    % \end{equation*}
    % with standard error
    % \begin{equation*}
    %     \operatorname{SE}\left( \overline{x}_1 - \overline{x}_2 \right) = \sqrt{\frac{\sigma_1^2}{n_1} + \frac{\sigma_2^2}{n_2}}
    % \end{equation*}
    % with is estimated by \(\sigma=s\).
    % The confidence interval for the difference of the two means is
    \begin{equation*}
        {CI}_{1-\alpha} = \overline{x}_1 - \overline{x}_2 \pm Z_{\alpha/2} \sqrt{\frac{s_1^2}{n_1} + \frac{s_2^2}{n_2}}.
    \end{equation*}
    for \(n_1,\: n_2 \geq 30\).
    \begin{equation*}
        T\left( \symbfit{x} \right) = \frac{\left( \overline{x}_1 - \overline{x}_2 \right) - \Delta_0}{\sqrt{\frac{s_1^2}{n_1} + \frac{s_2^2}{n_2}}}.
    \end{equation*}
    where \(\Delta_0 = \mu_1 - \mu_2\) is the hypothesised difference between the two population means.

    For small independent samples,
    \begin{equation*}
        T\left( \symbfit{x} \right) = \frac{\overline{x}_1 - \overline{x}_2}{\sqrt{s^2\left( \frac{1}{n_1} + \frac{1}{n_2} \right)}}.
    \end{equation*}
    If \(s_1^2 \ne s_2^2\) use
    \begin{equation*}
        s^2 \rightarrow s_p^2 = \frac{\left( n_1 - 1 \right)s_1^2 + \left( n_2 - 1 \right)s_2^2}{\nu}.
    \end{equation*}
    where \(\nu = n_1 + n_2 - 2\) for the two-sample \(t\)-test.
    If \(
    \frac{\max{\left( s_1^2,\: s_2^2 \right)}}{\min{\left( s_1^2,\: s_2^2 \right)}} > 3
    \), samples vary \textit{greatly}, use:
    \begin{equation*}
        T\left( \symbfit{x} \right) = \frac{\overline{x}_1 - \overline{x}_2 }{\sqrt{\frac{s_1^2}{n_1} + \frac{s_2^2}{n_2}}}
    \end{equation*}
    \begin{equation*}
        \nu = \floor*{\frac{\left( \frac{s_1^2}{n_1} + \frac{s_2^2}{n_2} \right)^2}{\frac{\left( s_1^2 / n_1 \right)^2}{n_1 - 1} + \frac{\left( s_2^2 / n_2 \right)^2}{n_2 - 1}}}
    \end{equation*}
    \subsection{Difference of Population Proportions}
    \begin{equation*}
        {CI}_{1-\alpha} = \begin{aligned}[t]
             & \hat{p}_1 - \hat{p}_2                                                                                                            \\
             & \pm Z_{\alpha/2} \sqrt{\tfrac{\hat{p}_1\left( 1 - \hat{p}_1 \right)}{n_1} + \tfrac{\hat{p}_2\left( 1 - \hat{p}_2 \right)}{n_2}}.
        \end{aligned}
    \end{equation*}
    \(n_i \hat{p}_i > 5\), and
    \(n_i \left( 1 - \hat{p}_i \right) > 5\) for \(i=1,2\).
    When the hypothesised difference is zero
    \begin{equation*}
        p_0 = \frac{x_1 + x_2}{n_1 + n_2} = \frac{\hat{p}_1 n_1 + \hat{p}_2 n_2}{n_1 + n_2}.
    \end{equation*}
    so that \(p_0 = p_1 = p_2\).
    \begin{equation*}
        T\left( \symbfit{x} \right) = \frac{\left( \hat{p}_1 - \hat{p}_2 \right)}{\sqrt{p_0 \left( 1 - p_0 \right) \left( \frac{1}{n_1} + \frac{1}{n_2} \right)}}.
    \end{equation*}
    % % unsure if needed
    When the hypothesised difference is not 0, i.e., \(p_1 - p_2 = \Delta_0\):
    \begin{equation*}
        T\left( \symbfit{x} \right) = \frac{\left( \hat{p}_1 - \hat{p}_2 \right) - \Delta_0}{\sqrt{\frac{\hat{p}_1 \left( 1 - \hat{p}_1 \right)}{n_1} + \frac{\hat{p}_2 \left( 1 - \hat{p}_2 \right)}{n_2}}}.
    \end{equation*}
    \subsection{Significance of Results}
    The test can only be used to reject the null hypothesis.
    When the strength against the null hypothesis is weak, we cannot assume that the null
    hypothesis is true, rather, the test is inconclusive and that there is no statistical significance.
    \subsection{Power}
    \begin{align*}
        \operatorname{Power} & = 1 - \beta                                                                                                   \\
                             & = 1 - \Pr{\left( \abs*{T\left( \symbfit{x} \right)} \leq Z_{\alpha/2} \,\vert\, \theta = \theta^\ast \right)} \\
                             & = \Pr{\left( \abs*{T\left( \symbfit{x} \right)} \geq Z_{\alpha/2} \,\vert\, \theta = \theta^\ast \right)}.
    \end{align*}
    For the true value \(\theta^\ast\) of \(\theta\) (not \(\theta_0\)).

    As \(n\) increases, the \(\beta\) decreases, and the power increases.
    \(n\) can therefore be selected to achieve a desired true value of \(\theta\) and a desired power.
    \subsection{P-Values}
    Rather than constructing \(R\) based on \(\alpha\), measure the strength of evidence
    against \(H_0\) using
    \begin{equation*}
        \alpha = \Pr{\left( Z \geq T\left( \symbfit{x} \right) \right)}.
    \end{equation*}
    The strength of evidence against \(H_0\) increases as the \(p\)-value decreases.
\end{multicols}
\hrule
\section{ANOVA}
Analyse effects of various factors that have more than two levels.
\begin{itemize}
    \item \textbf{Experimental unit} --- object whose response is measured, called the \textbf{dependent variable}.
    \item \textbf{Factor} --- independent variable controlled/varied in experiment. The \textbf{levels} are the values that the factor can take.
    \item \textbf{Treatment} --- specific combination of factor levels applied to an experimental unit.
    \item \textbf{Response} --- variable measured for each experimental unit.
\end{itemize}
The \(F\)-statistic \(F_{\mathrm{test}}\) (``F'' in ANOVA table) is the ratio of the mean squares for the treatment and error sources of variation,
or the ratio of the variation between treatments to the variation within treatments.
Reject \(H_0\) if \(\mathrm{MSXX} \gg \mathrm{MSE}\), (XX accounted for more of the total variance than the error).
\(F_{\mathrm{test}} \gg F_{\nu_1,\: \nu_2,\: \alpha}\),
\(\Pr{\left( F < F_{\nu_1,\: \nu_2,\: \alpha} \right)} = 1 - \alpha\).
\begin{multicols}{3}
    \subsection{One-Way ANOVA}
    Let the outcome of an experiment for replication \(j\) of treatment \(i\) be denoted by \(y_{ij}\).
    \begin{equation*}
        y_{ij} = \mu_i + \epsilon_{ij}
    \end{equation*}
    for mean treatment effect \(\mu_i\) and \(n = IJ\) experimental units.
    % The total variation in experimental outcomes can be described by the \textbf{total sum of squares} (SST):
    % \begin{equation*}
    %     \mathrm{SST} = \sum_{i = 1}^I \sum_{j = 1}^J \left( y_{ij} - \overline{y}_{..} \right)^2
    % \end{equation*}
    % where \(\overline{y}_{..}\) is the grand mean.
    % The total sum of squares can be decomposed into two parts, the \textbf{error sum of squares} (SSE),
    % which is the pooled variation in the outcomes within treatment group \(i\):
    % \begin{equation*}
    %     \mathrm{SSE} = \sum_{i = 1}^I \sum_{j = 1}^J \left( y_{ij} - \overline{y}_i \right)^2
    % \end{equation*}
    % and the \textbf{treatment sum of squares} (SSTr), which is the variation in the outcomes due to the different treatments:
    % \begin{equation*}
    %     \mathrm{SSTr} = J \sum_{i = 1}^I \left( \overline{y}_i - \overline{y}_{..} \right)^2
    % \end{equation*}
    % where \(\overline{y}_i\) is the mean response for treatment \(i\) given by
    % \begin{equation*}
    %     \overline{y}_i = \frac{1}{J} \sum_{j = 1}^J y_{ij}.
    % \end{equation*}
    Use the hypothesis:
    \begin{align*}
        H_0 & : \mu_i = \mu_j \quad \text{for all} \quad i \neq j \\
        H_A & : \text{at least one \(\mu_i\) is different}
    \end{align*}
    \subsection{Tukey's Honest Significant Difference Test}
    Tests the hypothesis that the mean of each treatment is equal to the mean of the other treatments.
    \begin{equation*}
        H_0 : \mu_i = \mu_j \quad \text{for all} \quad i \neq j
    \end{equation*}
    where we can reject individual pairs of treatment means if the \(p\)-value is less than the significance level.
    \subsection{Two-way ANOVA with Blocking}
    When variation in the responses arise due to factors other than the treatment factors.
    Uses \(I\) treatments and \(J > I\) blocks with \(I\) subjects to isolate block-to-block variability.
    \begin{equation*}
        y_{ij} = \alpha_i + \beta_j + \epsilon_{ij}
    \end{equation*}
    for \(\alpha_i\) and \(\beta_j\) the mean treatment effect and block effect, respectively.

    The null hypotheses are:
    \begin{align*}
        H_0 & : \alpha_i = \alpha_j \quad \text{for all} \quad i \neq j \\
        H_A & : \text{at least one \(\alpha_i\) is different}
    \end{align*}
    or
    \begin{align*}
        H_0 & : \beta_i = \beta_j \quad \text{for all} \quad i \neq j \\
        H_A & : \text{at least one \(\beta_i\) is different}
    \end{align*}
    \columnbreak
    \subsection{Two-Way ANOVA with Interaction}
    When both factors, and interactions between factors are of interest.
    \begin{equation*}
        y_{ijk} = \alpha_i + \beta_j + \left( \alpha \beta \right)_{ij} + \epsilon_{ijk}
    \end{equation*}
    where \(\alpha_i\) is the mean effect of the first factor, \(\beta_j\) is the mean effect of the second
    factor and \(\left( \alpha \beta \right)_{ij}\) is the mean effect of the interaction between the two factors.

    Factor \(A\) has \(I\) levels and factor \(B\) has \(J\) levels, and
    each of these factors is replicated \(K\) times.
    The null hypotheses are:
    \begin{align*}
        H_0 & : \alpha_i = \alpha_j \quad \text{for all} \quad i \neq j \\
        H_A & : \text{at least one \(\alpha_i\) is different}
    \end{align*}
    or
    \begin{align*}
        H_0 & : \beta_i = \beta_j \quad \text{for all} \quad i \neq j \\
        H_A & : \text{at least one \(\beta_i\) is different}
    \end{align*}
    or
    \begin{align*}
        H_0 & : \text{No interaction between A and B} \\
        H_A & : \text{A and B interact}.
    \end{align*}
\end{multicols}
\begin{multicols}{3}
    \section{Linear Regression}
    Relationship between a response variable \(y\) and a predictor \(x\):
    \begin{equation*}
        y_i = \beta_0 + \beta_1 x_i + \epsilon_i
    \end{equation*}
    where \(\epsilon_i\) is the residual,
    \begin{equation*}
        \epsilon_i \overset{\mathrm{iid}}{\sim} \mathrm{N}\left( 0,\: \sigma^2 \right).
    \end{equation*}
    \begin{equation*}
        \therefore \ y_i \sim \mathrm{N}\left( \beta_0 + \beta_1 x_i,\: \sigma^2 \right).
    \end{equation*}
    \subsection{Coefficients}
    \begin{align*}
        \hat{\beta}_0 & = \overline{y} - \hat{\beta}_1 \overline{x}                                                                                                     \sim \mathrm{N}\left( \beta_0,\: s^2_{\hat{\beta}_0} \right)  \\
        \hat{\beta}_1 & = \tfrac{\sum_{i = 1}^n \left( x_i - \overline{x} \right) \left( y_i - \overline{y} \right)}{\sum_{i = 1}^n \left( x_i - \overline{x} \right)^2} \sim \mathrm{N}\left( \beta_1,\: s^2_{\hat{\beta}_1} \right) \\
        s^2           & = \frac{1}{n - 2} \sum_{i = 1}^n \left( y_i - \hat{\beta}_0 - \hat{\beta}_1 x_i \right)^2
    \end{align*}
    \(s^2 = \frac{\mathrm{SSE}}{n - 2}\) from ANOVA\@.
    \begin{align*}
        s^2_{\hat{\beta}_0} & = \frac{s^2 \overline{x}^2}{n \sum_{i = 1}^n \left( x_i - \overline{x} \right)^2} \\
        s^2_{\hat{\beta}_1} & = \frac{s^2}{\sum_{i = 1}^n \left( x_i - \overline{x} \right)^2}
    \end{align*}
    \subsection{Hypothesis Testing}
    Confidence intervals:
    \begin{align*}
        \hat{\beta}_0 & \pm t_{\alpha / 2, n - 2} s_{\hat{\beta}_0} \\
        \hat{\beta}_1 & \pm t_{\alpha / 2, n - 2} s_{\hat{\beta}_1}
    \end{align*}
    where
    \begin{equation*}
        \Pr{\left( t < t_{\alpha / 2, n - 2} \right)} = 1 - \alpha / 2.
    \end{equation*}
    The null hypothesis for \(\beta_1\) tests whether there is indeed a linear relationship between the
    predictor and response variables:
    \begin{align*}
        H_0 & : \beta_1 = 0    \\
        H_A & : \beta_1 \neq 0
    \end{align*}
    with test statistic
    \begin{equation*}
        t_{\hat{\beta}_1} = \frac{\hat{\beta}_1 - \cancelto{0}{\beta_1}}{s_{\hat{\beta}_1}} \sim t_{n - 2}
    \end{equation*}
    \subsection{Assumptions}
    \begin{enumerate}
        \item The parameter estimates \(\hat{\beta}_0\) and \(\hat{\beta}_1\) are unbiased, i.e., the expected value of \(\epsilon_i = y_i - \hat{\beta}_0 - \hat{\beta}_1 x_i\) is zero.
        \item The residuals \(\epsilon_i\) are independent, i.e., \(\Corr{\left( \epsilon_i,\: \epsilon_j \right)} = 0\) for \(i \neq j\).
        \item The residuals follow a Gaussian distribution, i.e., \(\epsilon_i \sim \mathrm{N}\left( 0,\: \sigma^2 \right)\).
    \end{enumerate}
    Diagnostics to test assumptions:
    \begin{enumerate}
        \item Using a histogram of the residuals, we can check whether the residuals are unimodal and thus normally distributed.
        \item Using a \(q\)-\(q\) plot, we can check whether the residuals lie on a straight line.
        \item Using a plot of the residuals, we can check whether the residuals are independent, i.e.,
              there are no patterns in the residuals and their variance is roughly equal or constant (they lie randomly around 0).
              This is known as homoscedasticity.
    \end{enumerate}
    \subsection{ANOVA on Linear Regression}
    How much variation in \(y\) is explained by the model.
    The null hypothesis is that \textit{the model explains more variation in \(y\) than the sample mean \(\overline{y}\)},
    with the alternative explaining \textit{less}. For 1 independent variable, \(F \equiv t^2\), and the null hypothesis is simply
    \(H_0: \beta_1 = 0\).
    \subsection{Coefficient of Determination \texorpdfstring{\(R^2\)}{R2}}
    Proportion of the total variation in \(y\) that is explained by the model.
    \begin{equation*}
        R^2 = \frac{\mathrm{SSR}}{\mathrm{SST}} = 1 - \frac{\mathrm{SSE}}{\mathrm{SST}}
    \end{equation*}
    \(R^2\) is subjective: \(R^2 \approx 1\) is good but may indicate ``over-fitting''
    in the model, small values may also be acceptable.
    \subsection{Estimation and Prediction}
    Given the estimate \(\hat{y}_i = \hat{\beta}_0 + \hat{\beta}_1 x_i\):
    \begin{equation*}
        s_{\hat{y}_i} = \sqrt{s^2 \left( \frac{1}{n} + \frac{\left( x_i - \overline{x} \right)^2}{\sum_{i = 1}^n \left( x_i - \overline{x} \right)^2} \right)}
    \end{equation*}
    the confidence interval for the true value of \(y_i\) is
    \begin{equation*}
        \hat{y}_i \pm t_{n - 2, 1 - \alpha / 2} s_{\hat{y}_i}
    \end{equation*}
    based on the sampling distribution
    \begin{equation*}
        \frac{\hat{y}_i - \E{\left( y_i \right)}}{s_{\hat{y}_i}} \sim t_{n - 2}.
    \end{equation*}
    A predicted value \(y^\ast\) for an unobserved \(x^\ast\) gives \(y^\ast = \hat{\beta}_0 + \hat{\beta}_1 x^\ast\), so that:
    \begin{equation*}
        s_{y^\ast} = \sqrt{s^2 \left( 1 + \frac{1}{n} + \frac{\left( x^\ast - \overline{x} \right)^2}{\sum_{i = 1}^n \left( x_i - \overline{x} \right)^2} \right)}
    \end{equation*}
    which gives the prediction interval,
    \begin{equation*}
        y^\ast \pm t_{n - 2, 1 - \alpha / 2} s_{y^\ast}.
    \end{equation*}
    We are most confident when \(x\) is near \(\overline{x}\) (both bands are narrowest near \(\overline{x}\)),
    with the prediction interval being wider than the confidence interval. Caution should be used when extrapolating
    outside the data domain.
    \subsection{ANCOVA}
    Allows for the inclusion of continuous covariates that should be accounted for, but could not be controlled for.
    The ANOVA model is extended to
    \begin{equation*}
        y_{ij} = \alpha_i + \epsilon_{ij} \rightarrow \alpha_i + \beta\left( x_{ij} - \overline{x} \right) + \epsilon_{ij}.
    \end{equation*}
    where the ANCOVA model accounts for covariate values \(x_{ij}\).

    These terms increase the power of the test in detecting treatment effects.
    All constraints of linear regression regarding the independence, homogeneity, and normality
    of residuals also apply to the covariate effects in the model.

    This model also assumes that there is homogeneity in regression slopes,
    or \(\beta\) is approximately equal for all levels of treatment.
    This can be confirmed visually, or by fitting an interaction term between the treatment and covariates and testing
    if the interaction term is \textbf{not} significant (\(p > 0.05\)), indicating that it is the same for all treatments.
\end{multicols}
\hrule
\section{Categorical Data Analysis}
Data are counts of members in each category, and we are interested in the relationships between categories.
\begin{multicols}{3}
    \begingroup
    \allowdisplaybreaks
    \raggedbottom
    \subsection{\texorpdfstring{\(2 \times 2\)}{2x2} Contingency Tables}
    2 factors with 2 categories, where cells are the counts of observations in each category.
    Counts are random variables and have sampling distributions, but because row/column sums are fixed, they are not
    independent.
    \subsection{Hypergeometric Distribution}
    Probability of drawing \(k\) successes from a population of size \(N\) without replacement,
    where the population contains \(K\) successes, and the draw size is \(n\):
    \begin{equation*}
        \Pr{\left( X = k \right)} = \frac{\binom{K}{k} \binom{N - K}{n - k}}{\binom{N}{n}}.
    \end{equation*}
    Using the example table below,
    the probability of drawing \(k_{11}\) successes from the population of size \(N\) without replacement, where the population contains a total of
    \(K_1\) successes, and we draw a sample of size \(n_1\) is given by
    \begin{equation*}
        \Pr{\left( X = k_{11} \right)} = \frac{\binom{K_1}{k_{11}} \binom{N - K_1}{n_1 - k_{11}}}{\binom{N}{n_1}}.
    \end{equation*}
    We can construct a null hypothesis that tests whether the two factors have an equal probability of being
    associated with a category, then we can use the hypergeometric distribution to calculate the probability of
    observing the data given the null hypothesis, and reject the null hypothesis if the probability (two-sided multiplied by 2)
    is less than 0.05.
    \subsection{Chi-Squared Distribution}
    \begin{equation*}
        X = \sum_{i = 1}^n Z_i^2 \sim \chi_\nu^2
    \end{equation*}
    with \(Z_i \sim \mathrm{N}\left( 0,\: 1 \right)\) and \(\nu = n - 1\).
    \subsection{Test of Homogeneity}
    Tests whether the distribution
    of items across categories is the same for different factors.

    Using the table below,
    calculate the expected counts for each cell in the table as
    \begin{equation*}
        E_{ij} = \frac{n_{i.} n_{.j}}{n_{..}}.
    \end{equation*}
    The null hypothesis is that the distribution of items across
    categories is the same for all factors,
    \begin{align*}
        H_0 & : \pi_{ij} = \pi_{i} \forall i, j                                 \\
        H_1 & : \text{at least one \(\pi_{ij}\) is different from \(\pi_{i}\)}.
    \end{align*}
    \begin{equation*}
        \pi_i \approx \hat{\pi}_i = \frac{n_{i.}}{n_{..}}.
    \end{equation*}
    Assume \(\pi_{ij} \sim \operatorname{Poisson}\)
    then the \(Z\)-score \(Z_{ij}\) is given by
    \begin{equation*}
        Z_{ij} = \frac{\pi_{ij} - E_{ij}}{\sqrt{E_{ij}}} \sim \mathrm{N}\left( 0,\: 1 \right).
    \end{equation*}
    The \(\chi^2\) statistic is given by
    \begin{equation*}
        X^2 = \sum_{i = 1}^I \sum_{j = 1}^J \frac{\left( \pi_{ij} - E_{ij} \right)^2}{E_{ij}} \sim \chi_{\nu}^2
    \end{equation*}
    for \(\nu = \left( I - 1 \right)\left( J - 1 \right)\) degrees of freedom. We can therefore reject the null hypothesis if
    \begin{equation*}
        X^2 > \chi_{\nu, \: 1 - \alpha}^2.
    \end{equation*}
    This test assumes that row/column sums (\(\pi_{i.}\) and \(\pi_{.j}\)) are fixed.
    \subsection{Test of Independence}
    Tests whether the distribution of items across categories is independent of the
    factors:
    \begin{align*}
        H_0 & : \text{\parbox{4.5cm}{The categories and factors are independent}}            \\
        H_A & : \text{\parbox{4.5cm}{at least one category is not independent of a factor.}}
    \end{align*}
    We can reject the null hypothesis if
    \begin{equation*}
        X^2 > \chi_{\nu, \: 1 - \alpha}^2.
    \end{equation*}
    This test assumes only the total \(n\) is fixed.
    \subsection{Uniform Distribution}
    Single trial \(X\) in a set of equally likely elements.
    \subsection{Bernoulli (Binary) Distribution}
    Boolean-valued outcome \(X\), i.e., success (1) or failure (0).
    \subsection{Binomial Distribution}
    Number of successes \(X\) for \(n\) independent trials.
    \subsection{Geometric Distribution}
    Number of trials \(N\) up to and including the first success.
    \subsection{Poisson Distribution}
    Number of events \(N\) which occur over a fixed interval of time: \(\lambda = \eta t\).
    \subsection{Uniform Distribution}
    Outcome \(X\) within some interval, where the probability of an outcome in one interval is the same as all other intervals of the same length.
    \subsection{Exponential Distribution}
    Time \(T\) between events with rate \(\eta\).
    \subsection{Memoryless Property}
    % For \(T \sim \operatorname{Exp}{\left( \lambda \right)}\):
    \begin{equation*}
        \Pr{\left( T > s + t \,\vert\, T > t \right)} = \Pr{\left( T > s \right)}
    \end{equation*}
    \endgroup
\end{multicols}
\begin{multicols}{2}
    \begin{table}[H]
        \centering
        \begin{tabular}{ccc}
            \toprule
            \textbf{Equality}        & \textbf{Difference}            & \(R\)                                                 \\
            \midrule
            \(\theta = \theta_0\)    & \(\theta_1 - \theta_2 = 0\)    & \(\abs*{T\left( \symbfit{x} \right)} > Z_{\alpha/2}\) \\
            \(\theta \leq \theta_0\) & \(\theta_1 - \theta_2 \leq 0\) & \(T\left( \symbfit{x} \right) > Z_{\alpha}\)          \\
            \(\theta \geq \theta_0\) & \(\theta_1 - \theta_2 \geq 0\) & \(T\left( \symbfit{x} \right) < -Z_{\alpha}\)         \\
            \bottomrule
        \end{tabular}
        \caption{Rejection regions for hypothesis testing.}
        \begin{equation*}
            \Pr{\left( Z \geq Z_{\alpha / 2} \right)} = \frac{\alpha}{2},\: \Pr{\left( Z \geq Z_{\alpha} \right)} = \Pr{\left( Z \leq -Z_{\alpha} \right)} = \alpha
        \end{equation*}
    \end{table}
    \begin{table}[H]
        \centering
        \begin{tabular}{c|cc}
            \toprule
            \textbf{Decision} & \textbf{\(H_0\) true}     & \textbf{\(H_0\) false}    \\
            \midrule
            \textbf{Reject}   & \(\alpha\) (Type I error) & \(1 - \beta\) (Power)     \\
            \textbf{Fail}     & \(1 - \alpha\)            & \(\beta\) (Type II error) \\
            \bottomrule
        \end{tabular}
        \caption{Probability of rows given columns.}
    \end{table}
    % \begin{table}[H]
    %     \centering
    %     \begin{tabular}{c c c }
    %         \toprule
    %                                                  & \textbf{Discrete}                              & \textbf{Continuous}                                    \\
    %         \midrule
    %         \(\E{\left( X \right)}\)                 & \(\sum_{\Omega} xp_x\)                         & \(\int_{\Omega} xf(x)\odif{x}\)                        \\
    %         \(\E{\left( g\left( X \right) \right)}\) & \(\sum_{\Omega} g\left( x \right)p_x\)         & \(\int_{\Omega} g\left( x \right)f(x)\odif{x}\)        \\
    %         \(\Var{\left( X \right)}\)               & \(\sum_{\Omega} \left( x - \mu \right)^2 p_x\) & \(\int_{\Omega} \left( x - \mu \right)^2f(x)\odif{x}\) \\
    %         \bottomrule
    %     \end{tabular}
    %     \caption{Probability rules for univariate \(X\).} % \label{}
    % \end{table}
\end{multicols}
\begin{table}[H]
    \centering
    \begin{tabular}{c c c c c c}
        \toprule
        \textbf{Distribution}                                     & \textbf{Restrictions}                                  & \textbf{PMF}                                      & \textbf{CDF}                                                     & \(\E{\left( X \right)}\) & \(\Var{\left( X \right)}\)                    \\
        \midrule
        \(X \sim \operatorname{Uniform}{\left( a,\: b \right)}\)  & \(x \in \left\{ a, \dots, b \right\}\)                 & \(\frac{1}{b - a + 1}\)                           & \(\frac{x - a + 1}{b - a + 1}\)                                  & \(\frac{a + b}{2}\)      & \(\frac{\left( b - a + 1 \right)^2 - 1}{12}\) \\
        \(X \sim \operatorname{Bernoulli}{\left( p \right)}\)     & \(p \in \interval{0}{1}, x \in \left\{ 0, 1 \right\}\) & \(p^x \left( 1 - p \right)^{1 - x}\)              & \(1 - p\)                                                        & \(p\)                    & \(p \left( 1 - p \right)\)                    \\
        \(X \sim \operatorname{Binomial}{\left( n,\: p \right)}\) & \(x \in \left\{ 0, \dots, n \right\}\)                 & \(\binom{n}{x} p^x \left( 1 - p \right)^{n - x}\) & \(\sum_{u = 0}^x \binom{n}{u} p^u \left( 1 - p \right)^{n - u}\) & \(np\)                   & \(np\left( 1 - p \right)\)                    \\
        \(N \sim \operatorname{Geometric}{\left( p \right)}\)     & \(n \geq 1\)                                           & \(\left( 1 - p \right)^{n - 1} p\)                & \(1 - \left( 1 - p \right)^n\)                                   & \(\frac{1}{p}\)          & \(\frac{1 - p}{p^2}\)                         \\
        \(N \sim \operatorname{Poisson}{\left( \lambda \right)}\) & \(n \geq 0\)                                           & \(\frac{\lambda^n e^{-\lambda}}{n!}\)             & \(e^{-\lambda} \sum_{u = 0}^n \frac{\lambda^u}{u!}\)             & \(\lambda\)              & \(\lambda\)                                   \\
        \bottomrule
    \end{tabular}
    \caption{Discrete probability distributions.} % \label{}
\end{table}
\begin{table}[H]
    \centering
    \begin{tabular}{c c c c c c}
        \toprule
        \textbf{Distribution}                                       & \textbf{Restrictions} & \textbf{PDF}                                                                         & \textbf{CDF}                                                                            & \(\E{\left( X \right)}\) & \(\Var{\left( X \right)}\)            \\
        \midrule
        \(X \sim \operatorname{Uniform}{\left( a,\: b \right)}\)    & \(a < x < b\)         & \(\frac{1}{b - a}\)                                                                  & \(\frac{x - a}{b - a}\)                                                                 & \(\frac{a + b}{2}\)      & \(\frac{\left( b - a \right)^2}{12}\) \\
        \(T \sim \operatorname{Exp}{\left( \eta \right)}\)          & \(t > 0\)             & \(\eta e^{-\eta t}\)                                                                 & \(1 - e^{-\eta t}\)                                                                     & \(1/\eta\)               & \(1/\eta\)                            \\
        \(X \sim \operatorname{N}{\left( \mu,\: \sigma^2 \right)}\) & --                    & \(\frac{1}{\sqrt{2 \pi \sigma^2}} e^{-\frac{\left( x - \mu \right)^2}{2 \sigma^2}}\) & \(\frac{1}{2} \left( 1 + \erf{\left( \frac{x - \mu}{\sigma \sqrt{2}} \right)} \right)\) & \(\mu\)                  & \(\sigma^2\)                          \\
        \bottomrule
    \end{tabular}
    \caption{Continuous probability distributions.} % \label{}
\end{table}
\begin{table}[H]
    \centering
    \begin{tabular}{c c c c}
        \toprule
                                     & \textbf{Factor 1} & \textbf{Factor 2} & \textbf{Total Successes (\(K\))} \\
        \midrule
        \textbf{Category 1}          & \(k_{11}\)        & \(k_{12}\)        & \(K_1\)                          \\
        \textbf{Category 2}          & \(k_{21}\)        & \(k_{22}\)        & \(K_2\)                          \\
        \textbf{Total Draws (\(n\))} & \(n_1\)           & \(n_2\)           & \(N\)                            \\
        \bottomrule
    \end{tabular}
    \caption{Example \(2 \times 2\) contingency table.} % \label{}
\end{table}
\begin{table}[H]
    \centering
    \begin{tabular}{c c c c c}
        \toprule
                                        & \textbf{Factor 1} & \(\cdots\) & \textbf{Factor \(J\)} & \textbf{Total Successes (\(n_{i.}\))} \\
        \midrule
        \textbf{Category 1}             & \(\pi_{11}\)      & \(\cdots\) & \(\pi_{1J}\)          & \(n_{1.}\)                            \\
        \(\vdots\)                      & \(\vdots\)        & \(\ddots\) & \(\vdots\)            & \(\vdots\)                            \\
        \textbf{Category \(I\)}         & \(\pi_{I1}\)      & \(\cdots\) & \(\pi_{IJ}\)          & \(n_{..}\)                            \\
        \textbf{Total Draws \(n_{.j}\)} & \(n_{.1}\)        & \(\cdots\) & \(n_{.J}\)            & \(n_{..}\)                            \\
        \bottomrule
    \end{tabular}
    \caption{Example \(I \times J\) contingency table.} % \label{}
\end{table}
\end{document}
