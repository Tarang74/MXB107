%!TEX TS-program = xelatex
%!TEX options = -aux-directory=Debug -shell-escape -file-line-error -interaction=nonstopmode -halt-on-error -synctex=1 "%DOC%"
\documentclass{article}
\input{LaTeX-Submodule/template.tex}

% Additional packages & macros
\usepackage{mathdots}
\setitemize{leftmargin=*,topsep=1ex,partopsep=0ex,itemsep=-1ex,partopsep=0ex,parsep=1ex}
\setlist{leftmargin=*,topsep=1ex,partopsep=0ex,itemsep=-1ex,partopsep=0ex,parsep=1ex}

\usepackage{changepage} % Modify page width
\usepackage{multicol} % Use multiple columns
\usepackage[explicit]{titlesec} % Modify section heading styles

\titleformat{\section}{\raggedright\normalfont\bfseries}{}{0em}{#1}
\titleformat{\subsection}{\raggedright\normalfont\small\bfseries}{}{0em}{#1}

%% A4 page
\geometry{
a4paper,
margin = 10mm
}

%% Hide horizontal rule
\renewcommand{\headrulewidth}{0pt}
\fancyhead{}

%% Hide page numbers
\pagenumbering{gobble}

%% Multi-columns setup
\setlength\columnsep{4pt}

%% Paragraph setup
\setlength\parindent{0pt}
\setlength\parskip{0pt}

%% Customise section heading styles
% \titleformat*\section{\raggedright\bfseries}

\begin{document}
\begin{multicols}{3}
    \section{Events and Probability}
    \subsection{Event}
    Set of outcomes from an experiment.
    \subsection{Sample Space}
    Set of all possible outcomes \(\Omega\).
    \subsection{Intersection}
    Outcomes occur in both \(A\) and \(B\)
    \begin{equation*}
        A \cap B \quad\quad \text{or} \quad\quad AB
    \end{equation*}
    \subsection{Disjoint}
    No common outcomes, \(AB = \varnothing\)
    \begin{equation*}
        \Pr{\left( AB \right)} = \Pr{\left( A \,\vert\, B \right)} = 0
    \end{equation*}
    \subsection{Union}
    Set of outcomes in either \(A\) or \(B\)
    \begin{equation*}
        A \cup B
    \end{equation*}
    \subsection{Complement}
    Set of all outcomes not in \(A\), but in \(\Omega \)
    \begin{align*}
        A\overline{A}       & = \varnothing \\
        A \cup \overline{A} & = \Omega
    \end{align*}
    \subsection{Subset}
    \(A\) is a (non-strict) subset of \(B\) if all elements in \(A\) are also in \(B\) --- \(A \subset B\).
    \begin{equation*}
        AB = A \quad\quad \text{and} \quad\quad A \cup B = B
    \end{equation*}
    \begin{equation*}
        \forall A:A\subset \Omega \land \varnothing \subset A
    \end{equation*}
    \begin{align*}
        \Pr{\left( A \right)}             & \leq \Pr{\left( B \right)}                            \\
        \Pr{\left( B \,\vert\, A \right)} & = 1                                                   \\
        \Pr{\left( A \,\vert\, B \right)} & = \frac{\Pr{\left( A \right)}}{\Pr{\left( B \right)}}
    \end{align*}
    \subsection{Identities}
    \begin{align*}
        A \left( BC \right)            & = \left( AB \right) C                             \\
        A \cup \left( B \cup C \right) & = \left( A \cup B \right) \cup C                  \\
        A \left(B \cup C\right)        & = AB \cup AC                                      \\
        A \cup BC                      & = \left( A \cup B \right) \left( A \cup C \right)
    \end{align*}
    \subsection{Probability}
    Measure of the likeliness of an event occurring
    \begin{equation*}
        \Pr{\left( A \right)} \quad\quad \text{or} \quad\quad \mathrm{P}\left( A \right)
    \end{equation*}
    \begin{equation*}
        0 \leq \Pr{\left( A \right)} \leq 1
    \end{equation*}
    where a probability of 0 never happens, and 1 always happens.
    \begin{align*}
        \Pr{\left( \Omega \right)}       & = 1                         \\
        \Pr{\left( \overline{A} \right)} & = 1 - \Pr{\left( A \right)}
    \end{align*}
    \subsection{Multiplication Rule}
    For independent events \(A\) and \(B\)
    \begin{equation*}
        \Pr{\left( AB \right)} = \Pr{\left( A \right)} \Pr{\left( B \right)}.
    \end{equation*}
    For dependent events \(A\) and \(B\)
    \begin{equation*}
        \Pr{\left( AB \right)} = \Pr{\left( A \,\vert \, B \right)} \Pr{\left( B \right)}
    \end{equation*}
    \subsection{Addition Rule}
    For independent \(A\) and \(B\)
    \begin{equation*}
        \Pr{\left( A \cup B \right)} = \Pr{\left( A \right)} + \Pr{\left( B \right)} - \Pr{\left( AB \right)}.
    \end{equation*}
    If \(AB = \varnothing \), then \(\Pr{\left( AB \right)} = 0\), so that \(\Pr{\left( A \cup B \right)} = \Pr{\left( A \right)} + \Pr{\left( B \right)}\).
    \subsection{De Morgan's Laws}
    \begin{align*}
        \overline{A \cup B} & = \overline{A} \ \overline{B}     \\
        \overline{AB}       & = \overline{A} \cup \overline{B}.
    \end{align*}
    \begin{align*}
        \Pr{\left( A \cup B \right)} & = 1 - \Pr{\left( \overline{A} \ \overline{B} \right)}    \\
        \Pr{\left( AB \right)}       & = 1 - \Pr{\left( \overline{A} \cup \overline{B} \right)}
    \end{align*} 
    \subsection{Bayes' Theorem}
    \begin{equation*}
        \Pr{\left( A \,\vert\, B \right)} = \frac{\Pr{\left( B \,\vert\, A \right)}\Pr{\left( A \right)}}{\Pr{\left( B \right)}}
    \end{equation*}
    \section{Random Variables}
        Measurable variable whose value holds some uncertainty.
        An event is when a random variable assumes a certain value or range of values.
        \subsection{Probability Distribution}
        The probability distribution of a random variable \(X\) is a function that links all outcomes \(x \in \Omega\)
        to the probability that they will occur \(\Pr{\left( X = x \right)}\).
        \subsection{Probability Mass Function}
        \begin{equation*}
            \Pr{\left( X = x \right)} = p_x
        \end{equation*}
        \subsection{Probability Density Function}
        \begin{equation*}
            \Pr{\left( x_1 \leq X \leq x_2 \right)} = \int_{x_1}^{x_2} f\left( x \right) \odif{x}
        \end{equation*}
        \subsection{Cumulative Distribution Function}
        Probability that a random variable is
        less than or equal to a particular realisation \(x\).

        \(F\left( x \right)\) is a valid CDF if:
        \begin{enumerate}
            \item \(F\) is monotonically increasing and continuous
            \item \(\lim_{x \to -\infty} F\left( x \right) = 0\)
            \item \(\lim_{x \to \infty} F\left( x \right) = 1\)
        \end{enumerate}
        \begin{equation*}
            \odv{F\left( x \right)}{x} = \odv{}{x} \int_{-\infty}^x f\left( u \right) \odif{u} = f\left( x \right)
        \end{equation*}
        \subsection{Complementary CDF (Survival Function)}
        \begin{equation*}
            \Pr{\left( X > x \right)} = 1 - \Pr{\left( X \leq x \right)} = 1 - F\left( x \right)
        \end{equation*}
        \subsection{\texorpdfstring{\(p\)}{p}-Quantiles}
        \begin{equation*}
            F\left( x \right) = \int_{-\infty}^x f\left( u \right) \odif{u} = p
        \end{equation*}
        \subsection{Special Quantiles}
        \begin{align*}
            \text{Lower quartile \(q_1\):}  &  &  & p = \frac{1}{4} \\
            \text{Median \(m\):}            &  &  & p = \frac{1}{2} \\
            \text{Upper quartile \(q_2\):}  &  &  & p = \frac{3}{4} \\
            \text{Interquartile range IQR:} &  &  & q_2 - q_1
        \end{align*}
        \subsection{Quantile Function}
        \begin{equation*}
            x = F^{-1}\left( p \right) = Q\left( p \right)
        \end{equation*}
        \subsection{Expectation (Mean)}
        Expected value given an infinite number of observations. For \(a < c < b\):
        \begin{equation*}
            \E{\left(X\right)} = \begin{aligned}[t]
                 & -\int_{a}^c F\left( x \right) \odif{x}                     \\
                 & + \int_c^b \left(1 - F\left( x \right)\right) \odif{x} + c
            \end{aligned}
        \end{equation*}
        \subsection{Variance}
        Measure of spread of the distribution (average squared distance of each value from the mean).
        \begin{equation*}
            \Var{\left( X \right)} = \sigma^2 = \E{\left( X^2 \right)} - \E{\left( X \right)}^2
        \end{equation*}
        \subsection{Standard Deviation}
        \begin{equation*}
            \sigma = \sqrt{\Var{\left( X \right)}}
        \end{equation*}
\end{multicols}
\begin{table}[H]
    \centering
    \begin{tabular}{c c c c c c}
        \toprule
        \textbf{Distribution}                                      & \textbf{Restrictions}                                  & \textbf{PMF}                                              & \textbf{CDF}                                                             & \(\E{\left( X \right)}\)            & \(\Var{\left( X \right)}\)                    \\
        \midrule
        \(X \sim \operatorname{Uniform}{\left( a,\: b \right)}\)   & \(x \in \left\{ a, \dots, b \right\}\)                 & \(\frac{1}{b - a + 1}\)                                   & \(\frac{x - a + 1}{b - a + 1}\)                                          & \(\frac{a + b}{2}\)                 & \(\frac{\left( b - a + 1 \right)^2 - 1}{12}\) \\
        \(X \sim \operatorname{Bernoulli}{\left( p \right)}\)      & \(p \in \interval{0}{1}, x \in \left\{ 0, 1 \right\}\) & \(p^x \left( 1 - p \right)^{1 - x}\)                      & \(1 - p\)                                                                & \(p\)                               & \(p \left( 1 - p \right)\)                    \\
        \(X \sim \operatorname{Binomial}{\left( n,\: p \right)}\)  & \(x \in \left\{ 0, \dots, n \right\}\)                 & \(\binom{n}{x} p^x \left( 1 - p \right)^{n - x}\)         & \(\sum_{u = 0}^x \binom{n}{u} p^u \left( 1 - p \right)^{n - u}\)         & \(np\)                              & \(np\left( 1 - p \right)\)                    \\
        \( N \sim \operatorname{Poisson}{\left( \lambda \right)}\) & \(n \geq 0\)                                           & \(\frac{\lambda^n e^{-\lambda}}{n!}\)                     & \(e^{-\lambda} \sum_{u = 0}^n \frac{\lambda^u}{u!}\)                     & \(\lambda\)                         & \(\lambda\)                                   \\
        \bottomrule
    \end{tabular}
    \caption{Discrete probability distributions.} % \label{}
\end{table}
\begin{table}[H]
    \centering
    \begin{tabular}{c c c c c c}
        \toprule
        \textbf{Distribution}                                       & \textbf{Restrictions}                  & \textbf{PDF}                                                                         & \textbf{CDF}                                                                            & \(\E{\left( X \right)}\) & \(\Var{\left( X \right)}\)            \\
        \midrule
        \(X \sim \operatorname{Uniform}{\left( a,\: b \right)}\)    & \(a < x < b\)                          & \(\frac{1}{b - a}\)                                                                  & \(\frac{x - a}{b - a}\)                                                                 & \(\frac{a + b}{2}\)      & \(\frac{\left( b - a \right)^2}{12}\) \\
        \(T \sim \operatorname{Exp}{\left( \eta \right)}\)          & \(t > 0\)                              & \(\eta e^{-\eta t}\)                                                                 & \(1 - e^{-\eta t}\)                                                                     & \(1/\eta\)               & \(1/\eta\)                            \\
        \(X \sim \operatorname{N}{\left( \mu,\: \sigma^2 \right)}\) & \(x \in \left\{ 0, \dots, n \right\}\) & \(\frac{1}{\sqrt{2 \pi \sigma^2}} e^{-\frac{\left( x - \mu \right)^2}{2 \sigma^2}}\) & \(\frac{1}{2} \left( 1 + \erf{\left( \frac{x - \mu}{\sigma \sqrt{2}} \right)} \right)\) & \(\mu\)                  & \(\sigma^2\)                          \\
        \bottomrule
    \end{tabular}
    \caption{Continuous probability distributions.} % \label{}
\end{table}
\begin{minipage}{126.1962963mm}
    \begin{table}[H]
        \centering
        \begin{tabular}{c c c }
            \toprule
                                                     & \textbf{Discrete}                              & \textbf{Continuous}                                    \\
            \midrule
            Valid probabilities                      & \(0 \leq p_x \leq 1\)                          & \(f(x) \geq 0\)                                        \\
            Cumulative probability                   & \(\sum_{u \leq x} p_u\)                        & \(\int_{-\infty}^{x} f(u) \odif{u}\)                   \\
            \(\E{\left( X \right)}\)                 & \(\sum_{\Omega} xp_x\)                         & \(\int_{\Omega} xf(x)\odif{x}\)                        \\
            \(\E{\left( g\left( X \right) \right)}\) & \(\sum_{\Omega} g\left( x \right)p_x\)         & \(\int_{\Omega} g\left( x \right)f(x)\odif{x}\)        \\
            \(\Var{\left( X \right)}\)               & \(\sum_{\Omega} \left( x - \mu \right)^2 p_x\) & \(\int_{\Omega} \left( x - \mu \right)^2f(x)\odif{x}\) \\
            \bottomrule
        \end{tabular}
        \caption{Probability rules for univariate \(X\).} % \label{}
    \end{table}
\end{minipage}
\end{document}
